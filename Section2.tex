\documentclass{bxjsarticle}
\usepackage{zxjatype}
\usepackage[ipa]{zxjafont}
\usepackage{amsmath}
\usepackage{graphicx}
\usepackage{amssymb}

\title{PRML Section2 Exercise}
\author{Seitaro Shinagawa  品川 政太朗}
\date{\today}

\begin{document}
\maketitle

\begin{abstract}
表記はPRMLに準拠する。
\end{abstract}

\subsection*{演習2.1}
$p(x|\mu)=Bern(x|\mu)=\mu^x(1-\mu)^{1-x}$より、

\begin{align*}
\mathbb{E}[x]	&= 0 \cdot (1-\mu) + 1 \cdot \mu \\
				&= \mu
\end{align*}
\begin{align*}
var[x]	&= \mathbb{E}[(x-\mu)^2] \\
		&= \mathbb{E}[(x^2-2 \mu x+\mu^2)] \\
        &= \mathbb{E}[x^2]-2\mu^2+\mu^2 \\
        &= \mu-\mu^2 \\
        &= \mu(1-\mu)
\end{align*}
\begin{align*}
H[x]	&= -\sum_{x=0}^{1}{p(x|\mu)\ln p(x|\mu)} \\
		&= -(1-\mu)\ln (1-\mu)-\mu \ln \mu
\end{align*}

\subsection*{演習2.2}
$p(x=-1|\mu)+p(x=1|\mu)=1$より、この分布は正規化されているといえる。

\begin{align*}
\mathbb{E}[x]	&= -1 \cdot \frac{1-\mu}{2} + 1 \cdot \frac{1+\mu}{2} \\
				&= \mu
\end{align*}
\begin{align*}
var[x]	&= \mathbb{E}[x^2]-\mu^2 \\
        &= (-1)^2 \cdot \frac{1-\mu}{2} + 1^2 \cdot \frac{1+\mu}{2} -\mu^2 \\
        &= 1-\mu^2
\end{align*}
\begin{align*}
H[x]	&= -p(x=-1|\mu)\ln p(x=-1|\mu) - p(x=1|\mu)\ln p(x=1|\mu) \\
		&= -\frac{1+\mu}{2} \ln \frac{1+\mu}{2} - \frac{1-\mu}{2} \ln \frac{1-\mu}{2}
\end{align*}

\subsection*{演習2.3}
\begin{align*}
{}_N C_m+{}_N C_{m-1} &= \frac{N \cdot (N-1) \cdots (N-m+1)}{m \codt (m-1) \cdots 1} + \frac{N \cdot (N-1) \cdots (N-m+2)}{m-1 \cdot (m-2) \cdots 1} \\
                      &=\frac{\{N \cdot (N-1) \cdots (N-m+2)\}\{(N-m+1)+m\}}{m \cdot (m-1) \cdots 1} \\
                      &=\frac{N+1 \cdot (N) \cdots (N-m+2)}{m \cdot (m-1) \cdots 1} \\
                      &={}_{N+1} C_m
\end{align*}
N=1のとき、
\begin{align*}
\sum_{m=0}^N{\left(\begin{array}{ll} 1 \\ m \end{array} \right)x^m} &= \left(\begin{array}{ll} 1 \\ 0 \end{array} \right)x^0+\left(\begin{array}{ll} 1 \\ 1 \end{array} \right)x^1 \\
        &= (1+x)^1
\end{align*}
N=kのとき、
\[(1+x)^k = \sum_{m=0}^k{\left(\begin{array}{ll} k \\ m \end{array} \right)x^m}\]
上式が成り立つと仮定すると、
\begin{align*}
(1+x)^{k+1} &= (1+x)\sum_{m=0}^k{\left(\begin{array}{ll} k \\ m \end{array} \right)x^m} \\
            &= \sum_{m=0}^k{\Biggl\{\left(\begin{array}{ll} k \\ m \end{array} \right)x^m+\left(\begin{array}{ll} k \\ m \end{array} \right)x^{m+1}\Biggr\}} \\
            = \left(\begin{array}{ll} k \\ 0 \end{array} \right)x^0 &+ \left(\begin{array}{ll} k \\ 1 \end{array} \right)x^1 + \cdots + \left(\begin{array}{ll} k \\ k \end{array} \right)x^k \\
            &+ \left(\begin{array}{ll} k \\ 0 \end{array} \right)x^1 + \cdots + \left(\begin{array}{ll} k \\ k-1 \end{array} \right)x^k + \left(\begin{array}{ll} k \\ k \end{array} \right)x^{k+1} \\
            = \left(\begin{array}{ll} k+1 \\ 0 \end{array} \right)x^0 &+ \left(\begin{array}{ll} k+1 \\ 1 \end{array} \right)x^1 + \cdots + \left(\begin{array}{ll} k+1 \\ k \end{array} \right)x^k + \left(\begin{array}{ll} k+1 \\ k+1 \end{array} \right)x^{k+1} \\
            &= \sum_{m=0}^{k+1}{\left(\begin{array}{ll} k+1 \\ m \end{array} \right)x^m}
\end{align*}
よって、N=k+1のときも成り立つ。よって数学的帰納法により(2.263)式が証明された。\\
(2.264)式の左辺について、
\begin{align*}
\sum_{m=0}^N{\left(\begin{array}{ll} N \\ m \end{array} \right)\mu^m (1-\mu)^{N-m}} &= (1-\mu)^N \sum_{m=0}^N{\left(\begin{array}{ll} N \\ m \end{array} \right)(\frac{\mu}{1-\mu}})^N \\
                 &= (1-\mu)^N (1+\frac{\mu}{1-\mu})^N \\
                 &= 1
\end{align*}



\end{document}